\documentclass[a4paper,12pt,fleqn]{article} % добавить leqno в [] для нумерации слева

%%% Работа с русским языком
\usepackage{cmap}					% поиск в PDF
\usepackage{mathtext} 				% русские буквы в формулах
\usepackage[T2A]{fontenc}			% кодировка
\usepackage[utf8]{inputenc}			% кодировка исходного текста
\usepackage[english,russian]{babel}	% локализация и переносы

%%% Дополнительная работа с математикой
\usepackage{amsmath,amsfonts,amssymb,amsthm,mathtools} % AMS
\usepackage{icomma} % "Умная" запятая

%% Номера формул
%\mathtoolsset{showonlyrefs=true} % Показывать номера только у тех формул, на которые есть \eqref{} в тексте.
%\usepackage{leqno} %Нумерация формул слева

%% Шрифты
\usepackage{euscript}	 % Шрифт Евклид
\usepackage{mathrsfs} % Красивый матшрифт


%% Свои команды
\DeclareMathOperator{\sgn}{\mathop{sgn}}

%% Перенос знаков в формулах (по Львовскому)
\newcommand*{\hm}[1]{#1\nobreak\discretionary{}
{\hbox{$\mathsurround=0pt #1$}}{}}

%%% Заголовок
\author{}
\title{}
\date{}



%%% Теоремы
%\theoremstyle{definition}
%\newtheorem{case}{Задача}[section]
%\renewcommand{\thecase}{\arabic{case}}

%\theoremstyle{definition} % "Определение"
%\newtheorem{corollary}{Пункт}[case]
%\newtheorem{problem}{Задача}[section]
%\renewcommand{\thecorollary}{\asbuk{corollary}}

\usepackage{multicol} % Несколько колонок


%%% Работа с картинками
\usepackage{graphicx}  % Для вставки рисунков
\graphicspath{{images/}{images2/}}  % папки с картинками
\setlength\fboxsep{3pt} % Отступ рамки \fbox{} от рисунка
\setlength\fboxrule{1pt} % Толщина линий рамки \fbox{}
\usepackage{wrapfig} % Обтекание рисунков текстом

%%% Работа с таблицами
\usepackage{array,tabularx,tabulary,booktabs} % Дополнительная работа с таблицами
\usepackage{longtable}  % Длинные таблицы
\usepackage{multirow} % Слияние строк в таблице

%%% Страница
%\usepackage{extsizes} % Возможность сделать 14-й шрифт
\usepackage{geometry} % Простой способ задавать поля
\geometry{top=20mm}
\geometry{bottom=20mm}
\geometry{left=30mm}
\geometry{right=20mm}
%
\usepackage{fancyhdr} % Колонтитулы
\pagestyle{fancy}
\renewcommand{\headrulewidth}{0.1mm}  % Толщина линейки, отчеркивающей верхний колонтитул
%\lfoot{Нижний левый}
%\rfoot{Нижний правый}
\rhead{Глава \thesection}
%\chead{Верхний в центре}
%\lhead{}
% \cfoot{Нижний в центре} % По умолчанию здесь номер страницы

\usepackage{setspace} % Интерлиньяж
%\onehalfspacing % Интерлиньяж 1.5
%\doublespacing % Интерлиньяж 2
%\singlespacing % Интерлиньяж 1

\usepackage{soulutf8} % Модификаторы начертания
\usepackage{bm} % Модификатор начертания для математики

%\usepackage{tikz} % Работа с графикой
%\usepackage{pgfplots}
%\usepackage{pgfplotstable}
%\usepackage[utf8]{inputenc}
%\usetikzlibrary{arrows}


\usepackage{hyperref}
\usepackage[usenames,dvipsnames,svgnames,table,rgb]{xcolor}
\hypersetup{				% Гиперссылки
	unicode=true,           % русские буквы в раздела PDF
	pdftitle={Заголовок},   % Заголовок
	pdfauthor={Автор},      % Автор
	pdfsubject={Тема},      % Тема
	pdfcreator={Создатель}, % Создатель
	pdfproducer={Производитель}, % Производитель
	pdfkeywords={keyword1} {key2} {key3}, % Ключевые слова
	colorlinks=true,       	% false: ссылки в рамках; true: цветные ссылки
	linkcolor=yellow!55!red,          % внутренние ссылки
	citecolor=green,        % на библиографию
	filecolor=magenta,      % на файлы
	urlcolor=yellow          % на URL
} 
\usepackage{datetime}
\newdateformat{onlyyear}{\THEYEAR}
\newdateformat{defaultdate}{\THEDAY~\monthname[\THEMONTH] \THEYEAR~г.}

\usepackage{multicol} % Несколько колонок


\usepackage{tikz} % Работа с графикой
\usepackage{pgfplots} %взять данные из соседнего файла и построить по ним что-либо
\usepackage{pgfplotstable}
\usetikzlibrary{fadings}
\usetikzlibrary{decorations}
\usepgflibrary{decorations.pathmorphing}

\tikzfading[name=fade out, inner color=transparent!0,
outer color=transparent!100]

\usepackage[utf8]{inputenc}
\usetikzlibrary{arrows}
\usepackage{tcolorbox}
\usepackage{lipsum}
\tcbuselibrary{breakable}

%\usetikzlibrary{calc}

\usepackage{xparse}
\NewDocumentCommand{\definition}{mm}{ \textbf{Определение.} \textit{#1} --- #2 \vspace{0.3cm}}

\renewcommand{\familydefault}{\sfdefault}
\usepackage{float}

\begin{document} % конец преамбулы, начало документа
	
\tableofcontents

\newpage

\section{Угрозы в области финансовой безопасности}

\begin{tcolorbox}[colback=yellow!40!red!10!,colframe=yellow!40!red]


В результате изучения главы студент должен \textbf{знать} угрозы в сфере финансовой деятельности предприятия, угрозы нарушения порядка и правил валютного и экспортного контроля, противодействия отмыванию денежных средств, полученных преступным путем, контролю политически значимых лиц и финансированию терроризма; \textbf{уметь} систематизировать угрозы в сфере финансовой безопасности предприятия, связанные с правильностью ведения и достоверностью финансовой отчетности, правилами совершения валютных и экспортно-импортных операций, связанных с перемещением денежных средств и товаров через таможенную границу государства; \textbf{владеть} основными методами финансового анализа сделок и признаками выявления сомнительных операций. 
\end{tcolorbox}


\subsection{Риски и угрозы в области финансовой безопасности}


\definition{Финансы}{это система денежных отношений, связанная с формированием и использованием фондов денежных средств в процессе распределения и перераспределения валового внутреннего продукта.}

Финансы являются инструментом мобилизации средств хозяйствующими субъектами всех секторов экономики для осуществления экономической деятельности и органами государственной и муниципальной власти для реализации своих функций. Это комплекс финансовых операций, с помощью которых предприятия и органы власти аккумулируют денежные средства и осуществляют денежные расходы. Фонды аккумулируемых денежных средств в процессе их образования и движения выступают носителями финансовых отношений. Субъектами этих отношений являются юридические и физические лица, органы государственного и муниципального управления. \textit{Денежный оборот} является основой финансов, в связи с этим он отражает денежные отношения, складывающиеся в процессе распределительных и перераспределительных процессов, осуществляемых участниками экономической деятельности (см. рисунок \ref{berzon})\footnote{Финансы. Под редакцией Берзона Н.И. 2013. М. Юрайт, стр. 18} . При этом финансовые решения, принимаемые на всех уровнях, являются взаимосвязанными. От предприятий в сторону государства направлен финансовый поток в виде налогов и сборов. Именно за счет этих поступлений формируется доходная часть государственного бюджета. Со стороны государства на предприятия поступают денежные средства в виде дотаций и кредитов. В финансовые взаимоотношения с государством вступает и население. \\[-1.7cm]
\begin{center}
	\begin{figure}[h]
\begin{tikzpicture}
[line cap=round,line join=round,>=triangle 45,x=1.0cm,y=1.0cm, scale = 1]
\draw [yellow!55!red, fill=yellow!55!red!10!] (-1,0) rectangle (3,2);
\draw [yellow!55!red, fill=yellow!55!red!10!] (9,0) rectangle (13,2);
\draw [yellow!55!red, fill=yellow!55!red!10!] (4,8) rectangle (8,10);
\draw [yellow!55!red] (4,4) rectangle (8,6);
\draw[->] (3,0.5) -- (9,0.5);
\draw[->] (9,1.5) -- (3,1.5);
\draw[->] (0,2) -- (4,9.5);
\draw[->] (4.2,8) -- (1,2);
\draw[->] (12,2) -- (8,9.5);
\draw[->] (7.8,8) -- (11,2);
\draw[<->, dashed] (3,2) -- (4,4);
\draw[<->, dashed] (8,4) -- (9,2);
\draw[<->, dashed] (6,6) -- (6,8);
\node[draw, align=left,black, fill = white] at (6,0.5) {\scriptsize {выплаты процентов, дивидендов,}\\[-2mm]  \scriptsize {погашение облигаций}};
\node[draw, align=left,black, fill = white] at (6,1.5) {\scriptsize {приобретение ценных}\\[-2mm]  \scriptsize {бумаг предприятий}};
\node[draw, align=left,black, fill = white, rotate = 62] at (1.8,5.5) {\scriptsize {налоги, пошлины}\\[-2mm]  \scriptsize {и другие платежи}};
\node[draw, align=left,black, fill = white, rotate = 62] at (2.65,5.05) {\scriptsize {кредиты, дотации, финансирование}\\[-2mm]  \scriptsize {социально значимых проектов}};
\node[draw, align=left,black, fill = white, rotate = 298] at (10.2,5.5) {\scriptsize {подоходный налог, приобретение}\\[-2mm]  \scriptsize {государственных ценных бумаг}};
\node[draw, align=left,black, fill = white, rotate = 298] at (9.4,5) {\scriptsize {пенсии, стипендии,}\\[-2mm]  \scriptsize {социальные пособия}};
\node[align=left] at (6,5) {\Large{Финансовые}\\\Large{посредники}};
\node[align=left] at (1,1) {\Large{Предприятия}};
\node[align=left] at (11,1) {\Large{Население}};
\node[align=left] at (6,9) {\Large{Государство}};
\end{tikzpicture}
\caption{Взаимодействие экономических субъектов по Н.\:И.~Берзону}\label{berzon}
\end{figure}
\end{center} 

Во взаимоотношениях государства и населения решающую роль играют социальные факторы. Со стороны государства – это финансовый поток в виде пенсий, стипендий, социальных пособий и других платежей. За счет бюджетных средств государство выполняет свою социальную функцию, поддерживая тех, кто еще не работает (школьники, студенты), и тех, кто уже не работает (пенсионеры), а также тех, кто не может работать по состоянию здоровья (инвалиды). Финансовые взаимоотношения складываются также между населением и предприятиями. Это, прежде всего, приобретение населением ценных бумаг предприятия. Получение гражданами заработной платы относится не к финансовым, а к трудовым отношениям предприятия и его работников. В качестве \textit{финансовых посредников} выступают банки, страховые компании, инвестиционные фонды и другие финансовые институты, оказывающие услуги всем участникам финансового рынка. Финансовые посредники аккумулируют денежные средства, поступающие от тех экономических субъектов, у которых возникает относительный избыток данных средств, и через механизм финансового рынка передают эти средства субъектам, испытывающим недостаток в финансовых ресурсах.

\begin{tcolorbox}[colback=yellow!55!red!5!,colframe=yellow!55!red,enforce breakable,% use only breakable in the real world!
	pad at break=1mm, title=Кейс 1. Знание механизма работы платежной системы]
	
	В 2010 г. клиент крупного банка «Иванов» обратился в кредитную организацию с сообщением о том, что неизвестные лица сняли с его индивидуального счета крупную денежную сумму. Информация поступила клиенту в виде SMS-сообщения на его мобильный телефон. Служба безопасности банка установила, что в результате действий профессиональных мошенников средства со счета «Иванова» были перечислены в другой крупный банк, где за два дня до описываемого инцидента был открыт счет также на имя «Иванова», без его ведома.
	
	Службы безопасности обоих банков организовали оперативное взаимодействие, проинформировали органы внутренних дел о ситуации, подготовили операцию по захвату мошенников с поличным при попытке обналичивания похищенных денег. Они привлекли к сотрудничеству профильные подразделения обоих банков, ответственные за осуществление платежей. Работники операционного управления банка № 1 установили, что деньги все-еще находятся на корреспондентском счете их банка в Московском главном территориальном управлении (МГТУ) Банка России. Они будут списаны оттуда на корреспондентский счет банка № 2 только очередным платежным рейсом, который состоится в 00 час. 00 мин. На основании полученной информации банк № 1 официально обратился в МГТУ Банка России и средства клиента были возвращены на его счет. Злоумышленники были установлены и задержаны на следующий день. Расследование, проведенное органами внутренних дел, показало, что хищение денежных средств было запланировано организованной группой мошенников. При этом утечка конкретной информации о номере персонального счета клиента произошла по вине доверенных лиц из его близкого окружения, которые халатно отнеслись к обеспечению сохранности доверенной им информации.

	\begin{itemize}
		\item[{\color{yellow!55!red}\Huge {  $ ? $}} \quad]   Каким еще путем можно было 	вернуть средства  клиента на его счет?
	\end{itemize}	
	
	 \end{tcolorbox}



	\renewcommand{\theenumi}{\arabic{enumi}}
	\renewcommand{\labelenumi}{(\theenumi)}

Основные риски и угрозы в области финансовой безопасности связаны с: 
\begin{enumerate}
	\item умышленными или неумышленными нарушениями технологических процессов без образования состава правонарушения;
	\item совершением действий, влекущих за собой потерю деловой репутации;
	\item совершением правонарушений в области бухгалтерского учета, сфере совершения валютно-экспортных операций; 
	\item нарушения правил осуществления внутреннего контроля в целях противодействия легализации (отмывания) денежных средств, полученных преступным путем, и финансированию терроризма (ПОД/ФТ); 
	\item выводу из легального денежного оборота средств в целях их последующего нецелевого использования, присвоения и использования в иных противоправных целях.   
\end{enumerate}


\begin{tcolorbox}[colback=yellow!40!red!5!,colframe=yellow!40!red,enforce breakable,% use only breakable in the real world!
	pad at break=1mm, title=Вопросы и задания для самоконтроля]
	\begin{itemize}
		\item[{\color{yellow!55!red}\Huge { $ ? $}} ]  Дайте определение понятия «потеря деловой репутации предприятия» в рамках его защиты от угроз в области финансовой безопасности.
		\item[{\color{yellow!55!red}\Huge {  $ ? $}} ] Расскажите об угрозах, которые могут нанести ущерб предприятию вследствие нарушения правил ведения бухгалтерской отчетности.
		\item[{\color{yellow!55!red}\Huge {  $ ? $}} ] Сообщите об угрозах нарушения правил валютно-экспортных операций.
		\item[{\color{yellow!55!red}\Huge {  $ ? $}} ] Расскажите об операциях, подлежащих обязательному контролю, при осуществлении контроля в целях ПОД/ФТ.
		\item[{\color{yellow!55!red}\Huge {  $ ? $}} ] Расскажите об угрозе нарушения порядка и правил осуществления контроля в целях ПОД/ФТ субъектом данного вида деятельности.
		
	\end{itemize}		
\end{tcolorbox}







\end{document}